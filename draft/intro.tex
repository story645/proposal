\documentclass[letterpaper,onecolumn,titlepage]{Ythesis}
\usepackage[utf8]{inputenc}
\usepackage{tikz}
\usepackage{array,multirow}
\usepackage{subcaption}
\usepackage{subfiles}
\usepackage{url}
\usepackage{amsmath}
\usepackage{amssymb}
\usepackage{float}
\usepackage{diagbox}
\usepackage{graphicx}
\usepackage[backend=bibtex, style=numeric-comp]{biblatex}
\bibliography{glasslab_viz}



\title{Make any stupid plot you want}
\author{Hannah Aizenman}
\committee{Dr. Michael Grossberg (Advisor), Dr. Robert Haralick, Dr. Lev Manovich, Dr. Huy Vo}
\submitted{}
\abstract{}
\begin{document}
\makefrontmatter

\section{Introduction}
\label{sec:introduction}
\begin{figure}
    \includegraphics[width=.5\textwidth]{figures/intro/viz_same.png}
    \caption[]{Implicit in visualization is the assumption that these three representations of data are equivalent, specifically that the measurements within a variable and relations of the measurements of each variable are preserved. }
    \label{fig:viz_same}
\end{figure}

\subsection{Thesis statement}
We define a visualization as a transform from data to graphic that preserves the topology of the data and the properties of the measurement type. In fig~\ref{fig:viz_same}, we implicitly assume that the translation from table to heatmap has preserved the order of observations (the rows) and that the perceptually uniform sequential colormap has been applied such that the ordering relation on floats matches the ordering on the colormap (darker colors map to larger numbers). We also make this assumption about color in the scatter map, and that the translation to size and position on screen also respect the ordering on floats. In this work, we propose to mathematically describe the transform of data to visual space such that we can make explicit the implicit topology and types visualizations preserve. We then propose a new architecture for the Python visualization library Matplotlib \cite{hunterMatplotlib2DGraphics2007} based on these descriptions because the Matplolib artist layer is analogous to the transforms. 


\section{Notation \& Definitions}
\label{sec:hunterMatplotlib2DGraphics2007}
System of transforms:
input: data w/ structure on fiber bundle + measurement type
artist: data -> visualization
(what structure of input is preserved in output)
we can provide list of
artist : topology it preserves
point/line/face parameters: measurement type properties 

ex: 
Line2D preserves continuity
Linestyle: categorical 

%%define basic properties of measurement of types in a theoretical way or C&P from a paper

Minimum assumption a transform will make - is this a valid transform for the slot? 

Set of questions/rules to apply such that see a new artist can be defined which lets us maybe simplify optimizations.  

M, Fiber Bundle, Section, operatons on \{data, visual\}, transform data space \& visual space,  

topology gives us that m is preserved
spivak gives us that v is preserved


temperature on globe:
\begin{equation}
data = [[t1], [t2_{la1, lo2}]
fiber = {lat, lon, temp}
m = the data is 2d continous
\end{equation}
artist has implict m
aesthetic paremeters transform in fiber that preserves type operations \& relations in the fiber
lat - relative distance + ordering 

\subsubsection{Graphical Elements}
There is a set of transform functions $T$ that maps from the data space $D$ to the visual space composed of geometric marks and aesthetic channels $V$ \cite{bertinIIPropertiesGraphic2011, munznerMarksChannels}.  We propose that a visualization is 
\begin{equation}
    T: D \rightarrow V
\end{equation}

Wilkenson proposes  that visualization is  \cite{wilkinsonGrammarGraphics2005, wilkinsonMathematicalFoundationAnalytic2010}

\begin{align}
\label{eq:gog_data_range}
G &= {(x, f(x)): x \in R and f(x)=e^{-x^2}}\\
F &= [-3,3] x [0,1]
\end{align}

\begin{equation}
\label{eq:gog_aesthetic_mapping}
A: x \mapsto x_{position}, f(x) \mapsto y_{position}
\end{equation}

\begin{equation}
G_{A} = A(F \cap G) 
\end{equation}

Wilkenson formulates plotting in terms of varsets, which are sets of variables where variables are:
\begin{equation}
V:O\mapsto V    
\end{equation}
Wilkenson's data algebra is equivalent to data reshapes and join and therefore \cite{wickhamLayeredGrammarGraphics2010,wilkinsonGrammarGraphics2005}


\subsection{What is the artist}
Visualization is a two step process where the artists $A$ transforms the data into visual encodings $V$ and then the renderer $R$ transfroms the encodings $V$ to the set of pixels in $P$.
\begin{equation}
    \label{eq:artist}
    A: D \mapsto M
\end{equation}
TYPES: A(D) = M
Spivak A is analogous DT

D is the type  which means it's indexexer K(M/indexer) + V (fiber bundle)
d is a section of a bundle is the data is the values given keys on m + vars on V (c in C)
a(d) = M

$M$ is a composition of the visual channels applied as ....
Is also a CW/simplicial complex/simplacial set 

Channels are ducktyping the variable types
Marks are encoding seperability 

Marks are mapping of simplacial complex into RGB space

(find paper on complex marks)

for every key value in M, have coordinates in the fiber (space)
section: keys -> values 

Scatter Plot:
K - 0D disconnected points
\begin{multline}
A: (C1, C2, C3...)\mapsto (M....)\\
t \in A
\end{multline}

-> preserve type operation and relation

M is the disjoint collection of the output of all these transformers where the transformers preserve the relation and order. 

Line Plot:
L - 1D, can be a function on an interval [a,b]
functions in the column...f(key) -> values

invariance: The symmetry group $A$ preserves invariance when the transforms $a \in A$ are symmetric such that $operation(column) = operation(channel)$ and $operation on data -> operation (mark)$

K might have symmetry -> move back and forth in time, preserved in K
symmatery in fibers -> scaled up/down, categorical,
preserved in the visual variables

groups on K and fibers and subgroups w/ different types and temperatures (Steven's fundemental data types-measurment scales and types)

intervals can transalte, ratio can only scale
group of fibers has to do w/ measurment types
channels themselves independent of data have their own symmatery

(make table of what symmateries are preserved)
visual transformation groups

instead of inputting parameter into input, get to all the other parameters - base thing is a square, 
preserved: operation(D) <-> operation (M)
if A(g*d) = A(d)*g

normalizations of scales in mpl - devision usually leaves same picture but axis is affected - visually length got uniformally smaller but we changed ticks to compensate
Axes has rules about how transform affects it - inside graph invariance is preserved

groups:
symmteries of K
symmetries of cartesian product of columns
each visual channel has symmtery group associated w/
orbit- space of possible visualiztions by changing the constants of each aesthetic (or $f_{red} \rightarrow f_{green}$)  
visual type might be orbit of that (orbit of function in space)->single function that can be transformed in that way 

each visualization type is an orbit \& then try to set up equivalence on these...

artist will respect that w/ equivariance, knob on one side will tweak knob on another



The aesthetic transfroms $a \in A$ are a symmetry group that preserves the group structure of the measurment scales as follows: 

categorical \& shapres have the group structure and so can do a $a: c \mapsto s$
Optimal transform $a_{i}$ is one from a measurment space to a visual channel where both sides have the same group transforms-
because equivaraince requires having the same transforms on the same side. 

artists take in class of visuals + 

A - matplotlib artists 
$A: D \mapsto CW$

$\tau: fb \mapsto vv $visual variable transformers
$\{\tau_{1}, \tau_{2}...\tau_{n}\}(K)\mapsto \{CW_{1}, CW_{2}, ...CW_{n}\}$ 
$\mapsto$ is what embeds the topology k 
functions that know how to embed the k (mark)
functions that work on the fibers $\tau$

%% distance along edge to transformed physical space, s function 
%% colormap_nrom - t: d \mapsto RGBA, continouts t: d(s) \mapsto 
%% 1d has to be parameterized  t: d(s), really want t: d\ mapsto RGBA
%%  d(s) \mapsto {d....} on interval [s_{0}..s_{n}]
%% flip this to the otherside so it's an adjacency matrix
%% how do we represent invariance/equivariance  A: D \mapsto \I
%% D - columns + topology (section + topology)
%% artist is the visual idiom 
%% {D cartesian product Tau} \mapsto I
%% Artist already has everything in VV space and composing into idiom
%% functions on simplex crossed wuth transformers - everything in visual space w/ same topology
%% A: visual \mapsto idiom
%% Artist (raw data + transforms )
%% carry topology over from VV space to I space -> singular CW complex 
%% (has no inherent structure) 
%% invaraince - shuffle_sq on tick labels = shuffle_sq on data 
%% how do you preserve invariance on topology->Idiom?
%% \tau_{2} is a new \tau function taht's really tau + 2
%% taus is a group of transform functions under composition 
%% tau gos data to visual variable
%% group of functions + 
%% measurement space groups is what class of functions don't destroy your data
%% \tau: d \mapsto vv - group of mapto functions
%% set of allowable \tau is where symmtery d/v are preserved 
%% d\v isomorphisms where tau takes you from one to another
%% d\v is  isomorphisms under group of taus where tau is the permutation group
%%
%% math: structure is same as type
Steven's measurement scales have the following structure \cite{stevensTheoryScalesMeasurement1946}:
%the big point of disagreement on the scales seems to be the stats column that we don't care about
\begin{table}
    \lable{tab:measument_type}
    \begin{tabular}{l l l l}
        nominal & permutation group & one to one substitutions & $=, \neq $\\
        ordinal & isotonic group & monotonic increasing functions &  $=, \neq $, $\leq, \geq$\\
        interval & general linear group & matrix operations &  $=, \neq, \leq, \geq, +, -, $\\
        ratio & similarity group & $ \mathbb{Z}^+ $ & 
    \end{tabular}
\end{table}

%%% make a second table later of what MPL currently supports
%%% redraw as graph
%%% set of all possible taus (operations can possibly be defined on both sides)

Split is discrete/ vs continous measurment types 
It is possible to write taus to ....

A: (v...v) -> I
%% tau is when we have data/visual lineups for labeling colorbar \ legend
%% colorbar vs. legend - continous vs discrete

\begin{table}
    \label{tab:taus}
    \begin{tabular}{l*{6}{c}} 
                              & position      &  size        & shape        & texture     & color \\
    nominal                   &  1            &  1           & 1            & 1           & 1     \\ 
    ordinal                   &  1            &  1           & 1            & 1           & 1     \\
    interval                  &  1            &  1           &              &             & 1     \\   
    ratio                     &  1            &  1           &              &             & 1     \\  
    \end{tabular}
\caption{Group of $\{t_{1}....t_{n}\}$ where $tau: d \mapsto v$ preserving equivariance such that $d$ and $v$ are isomorphic. $d$ is data of type \{nominal, ordinal, interval, ratio\} and $v$ are the visual channels. \cite{bertinIIPropertiesGraphic2011,munznerMarksChannels}}
\end{table}
 

Table~\ref{tab:taus} shows under which $\tau: d maps to v$, $d$ and $v$ can be isomorphisms.


%%% draw what the function graph looks like for scatter versus linesg
Scatter Plot:
K - 0D disconnected points
\begin{multline}
A: (C1, C2, C3...)\mapsto (M....)\\
t \in A 
\end{multline}

-> preserve type operation and relation

M is the disjoint collection of the output of all these transformers where the transformers preserve the relation and order. M is a bunch of disconnected Marks

Line Plot:
L - 1D, can be a function on an interval [a,b]
functions in the column...f(key) -> values

Box chart: area + scatter 


Combining 
\begin{equation}
    \label{eg:renderer}
    R: M \mapsto P
\end{equation}
Where $P$ is the set of pixels rendered to screen ${p_{i,j} \in P}$


An artist is a function F from X to Y
that satisfyies properties P1, ... PN
Step 1: Define X
step 2: Define Y
Spefiy Pi in terms of equations.


\printbibliography
\end{document}