The difference between drawing and data visualization is that visualization implies that some structure of the data is preserved in the visual presentation.* We formalize what structure is preserved and how via a mathematical topological artist model. In our model, data and graphics can be viewed as sections of fiber bundles; this allows for (1) decomposing the translation of data fields (variables) into visual channels via an equivariant map on the fibers and (2) a topology-preserving map of the base spaces that encode the preserving the dataset connectivity into graphical elements. Furthermore, this model supports an algebraic sum operation such that more complex visualizations can be built from simple ones. We illustrate the application of the model through case studies of a scatter plot, line plot, and heatmap. We show that this model can be implemented with a small prototype.

We call the function mapping sections of the data bundle to the graphical bundle the "Artist" based on analogous architecture elements in the Python visualization library Matplotlib. To demonstrate the practical value of the model, we propose a model driven re-architecture of the artist layer of the Matplotlib. We represent the topological base spaces using triangulation, make use of programming types for the fiber, and build on Matplotlib's existing infrastructure for the rendering. In addition to providing a way to ensure the library preserves structure, the functional decomposition of the artist in the model could improve modularity, maintainability, and point to ways in which the library could better support concurrency and interactivity. The thesis will follow through on this proposal to explore how to further develop the model, showing how it can support Matplotlib's current diverse range of data visualizations while providing a better platform for domain-specific visualization library developers.
