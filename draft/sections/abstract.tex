This work presents a functional model of the structure-preserving maps from data to visual representation to guide the development of visualization libraries. Our model, which we call the  topological equivariant artist model (TEAM), provides a means to express the constraints of preserving the data continuity in the graphic and faithfully translating the properties of the data variables into visual variables. We formalize these transformations as actions on sections of topological fiber bundles, which are mathematical structures that allow us to encode continuity as a base space, variable properties as a fiber space, and data as binding maps, called sections, between the base and fiber spaces. This abstraction allows us to generalize to any type of data structure, rather than assuming, for example, that the data is a relational table, image, data cube, or network-graph. Moreover, we extend the fiber bundle abstraction to the graphic objects that the data is mapped to. By doing so, we can track the preservation of data continuity in terms of continuous maps from the base space of the data bundle to the base space of the graphic bundle. Equivariant maps on the fiber spaces preserve the structure of the variables; this structure can be represented in terms of monoid actions, which are a generalization of the mathematical structure of Stevens’ theory of measurement scales. We briefly sketch that these transformations have an algebraic structure which lets us build complex components for visualization from simple ones. We demonstrate the utility of this model through case studies of a scatter plot, line plot, and image. To demonstrate the feasibility of the model, we implement a prototype of a scatter and line plot in the context of the Matplotlib Python visualization library.  We propose that the functional architecture derived from a TEAM based design specification can provide a basis for a more consistent API and better modularity, extendability, scaling and support for concurrency.