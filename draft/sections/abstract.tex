 A critical aspect of data visualization is that the graphical representation of data match the properties of the data; this fails when order is not preserved in representations of ordinal data or scale for numerical data. In this work, we propose that the mathematical notion of equivariance formalizes the expectation that graphics match the data. We developed a topological model in which data and graphics can be viewed as sections of fiber bundles. This model allows for (1) decomposing the translation of data fields (variables) into visual channels via an equivariant map on the fibers and (2) a topology-preserving map of the base spaces that translates the dataset connectivity into graphical elements. Furthermore, our model supports an algebraic sum operation such that more complex visualizations can be built from simple ones. We illustrate the application of the model through case studies of a scatter plot, line plot, and heatmap. We show that this model can be implemented with a small prototype.

 To demonstrate the practical value of our model, we propose a model driven re-architecture of the artist layer of Matplotlib. We represent the topological base spaces using triangulation, make use of programming types for the fiber, and build on Matplotlib's existing infrastructure for the rendering. In addition to providing a way to ensure the library preserves structure, the functional decomposition of the artist in the model could improve modularity, maintainability, and point to ways in which the library could better support concurrency and interactivity. The thesis will follow through on this proposal to explore how to further develop our model, showing how it can support Matplotlib's current diverse range of data visualizations while providing a better platform for domain-specific visualization library developers.
