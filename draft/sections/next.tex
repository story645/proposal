\documentclass[../main.tex]{subfiles}

\begin{document}
\section{Discussion}

\subsection{What have we done}
presented mathematical framework w/ topological models for the things we have done. 
 \begin{itemize}
    \item fiberbundle -> continuity + variables
    \item fiberbundle -> graphic continuity + display abstraction
    \item formalizing equivariance on both sides 
    \item prototype for scatter, line, bar
    \item simple composition into multibar
    \item commonly used container - dataframe
 \end{itemize}

\subsection{What are the limitations?}
This framework is aimed at developers, not end users which is why it's equation soup only understood by physicits and mathematicians.  

.  It is a way to describe the pathway from data to graphic in a 

\subsection{What else needs to be implemented?}
\begin{itemize}
    \item heatmaps/ proof of concept of face simplex
    \item composition algebra
    \item boxplot - multiglyph composition
    \item debatable: non-trivial bundle like data on mobius backend
    \item interactivity 
    \item concurrent artist
    \item numpy wrapper
\end{itemize}
\begin{itemize}
    \item provisional implementation using new architecture in Matplotlib
    \item developer-aimed documentation (API + narrative)
    \item proof of concept 3rd party user-facing library with a biology partner
    \item academic paper about building visualization tools on top of the core architecture
\end{itemize}

Other things that may or may not play a big role would be working out the details of the S->K interaction for dashboards. By this I particularly mean when S is a composite object spanning all the figures shown in the dashboard and a component of K may have different compoents of S mapping into it slicing it in different ways.
\subsection{What are the lessons learned?}

\end{document}

