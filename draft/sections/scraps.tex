s demonstrate that this framework of artists (functor between sheafs) works for all or possibly almost all the pyplot module as a composition of primitive artists of this framework. It's kind of like your table but we will need to walk through Q carefully and look at how sampling will function in quiver plots,
and contour plots. You will absolutely not have to have this worked out for the proposal, I'm just saying, composition, and demonstration of feasibility for the rest of pyplot will be important.
You will have the code for many/some of the primitives (eg scatter, line, bar, imshow, spine), and a description of the composition operator (which is a union/join/+). You will also need to realize K as a triangulated space (possibly CW-Complex) for the purposes of the computation.

The fun part for you can go through the ideas in Munsner, Bertin, Heer, Wilkonson interpreting these things in terms of the prior work. That sounds to me like a bookchapter. We will have to see if there is an opertunity to get that sent somewhere. You won't have much time for this for you proposal but if we have published a coherent model, this could be a nice desert.

Other things that may or may not play a big role would be working out the details of the S->K interaction for dashboards. By this I particularly mean when S is a composite object spanning all the figures shown in the dashboard and a component of K may have different compoents of S mapping into it slicing it in different ways.


*********************
One thought I had that might be a blocker we hit several times, and we already discussed this, is the difference between the object, and how it is represented. The number 6 is called "perfect" because it is the sum of its proper divisors 1,2, and 3. If we wrote 6 in binary, 110, it would still be the sum of its proper divisors, 1, 10, 11. "6" as a postive number is independent of how you represent it in binary, hex, octal or whatever. The fact 111 has 3 digits which are "1" is no a property of the number unless you add in the representation in base 10. In may (and physics and other sciences) we are constantly keeping track of what things are artifacts of our implementation, and what is invariant to ANY implementation. In CS I often run into students getting this confused. If you don't implement in a very particular way, constrained by the math, you end up breaking the invariance. 