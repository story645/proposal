\documentclass[../main.tex]{subfiles}
\begin{document}
\subsubsection{Marks}
Bertin describes a location on the plane as the signifying characteristic of a point, measurable length as the signifying characteristic of a line, and measurable size as the signifying characteristic of an area and that in display (pixel) space are the point, line, and area marks \cite{bertinIIPropertiesGraphic2011,carpendaleVisualRepresentationSemiology}. For each region $s$ in the display space $H$, the mark it belongs to can be found by mapping $s$ back to $K$ via the lookup on $S$ then taking $\xi(s)$ back to a point on $k \in K$ which lies on the connected component $J \subset K$. 
\begin{equation}
\begin{tikzcd}
    H \arrow[r, shift left] & S \arrow[l, "\rho(\xi^{-1}(J))", shift left] \arrow[rr, "\xi(s)", shift left] &  & J_{k} =  \{j \in K| \exists \Gamma \text{ s.t. } \Gamma(0)=k \text{ and }\Gamma(1)=j\} \arrow[ll, "\xi^{-1}(J)", shift left]
\end{tikzcd}
\label{eq:mark}
\end{equation}
To got back to the display space $H$  from the simplacial complex $J$ of the signifier implanted in the mark, the inverse image of $J \in S, \xi^{-1}(J)$ is pushed back to $S$, and then  $\rho(\xi^{-1}(J))$ maps it into $R^{7}$. 


\subsubsection{Visual Idioms: Equivalance class of artists}
n $O(E)$ of the same type, they output the same type of prerender $O(H)$:


Natural transformation + composition is partial ordering? Back and forth is equivalent 
\end{document}