\documentclass[../main.tex]{subfiles}
\begin{document}

\section{Notation \& Definitions}
In this section we introduce a mathematical description of the visualization pipeline where artist $A$ functions transform data of type $\Gamma(E)$ to an intermediate representation in prerendered display space of type $\Gamma(H)$:

\begin{equation}
    A: O(E) \rightarrow O(H)
    \label{eq:artist}
\end{equation}

\begin{equation}
    A: \tau \rightarrow \rho
\end{equation}

\begin{itemize}
\item $A$ is the function that converts an instance of data $\Gamma(E)$ to an instance of a visual representation $\Gamma(H)$ 
\item $E$ is a locally trivial fiber bundle over $K$ representing data space.
\item $K$ is a triangulizable space encoding the connectivity of the observations in the data. 
\item $H$ is a fiber bundle over $S$ representing visual space
\item $S$ is a simplacial complex of triangles encoding the connectivity of the visualization of the data in $E$
\item $\tau: K\rightarrow E$ is the data being visualized
\item $\rho: S \rightarrow H$ is the render map
\end{itemize}

When $E$ is a trivial fiber bundle $E = F \times K$, it can be assumed that all fibers $F_{k}$ over $k \in K$ are equal. Fiber bundles are product spaces of toplological spaces, which are a set of points with a set of neighborhoods for each point\cite{FiberBundle2020, rowlandFiberBundle}.

\subsection{Data Model}

We use a fiber bundle model to represent the data, as proposed by Butler 
\cite{butlerVectorBundleClassesForm1992,butlerVisualizationModelBased1989}. A fiber bundle is a structure $(E, K, \pi, F)$  consisting of topological spaces $E, K, F$ and the map from total space to base space:

\begin{equation}
    \begin{tikzcd}
        F \arrow[r, hook] & E \arrow[d, "\pi" description, bend right ] \\
                        & K \arrow[u, description, bend right]
    \end{tikzcd}
\end{equation}

where there is a bijection from $F$ to every fiber $F_k$ over point $k \in K$ in $E$ and the function $\pi: E \rightarrow K$ is the map into the $K$ quotient space of $E$. Every point in the base space $k \in K$ has a local open set neighborhood $U$ \cite{FiberBundle2020, rowlandFiberBundle}

\begin{equation}
    \begin{tikzcd}
        \pi^{-1}(U) \arrow[r, "\varphi" description] \arrow[d, "\pi" description] & U \times F \arrow[ld, "\mathrm{proj}_U"] \\
        U                                                                         &                                         
    \end{tikzcd}
    \label{eq:local_trivial}
\end{equation}
such that $\varphi: \pi^{-1}(U) \rightarrow U \times F$ is a homeomorphism where $\pi$ and $\mathrm{proj}_U$ both map to $U$ and the fiber over $k$ $F_k = \pi^{-1}({k \in K}) $ is homomorphic to the fiber $F$.

The section $\tau$ is the mapping $\tau: K\rightarrow E$ 
\begin{equation}
    \begin{tikzcd}
        F \arrow[r, hook] & E \arrow[d, "\pi" description, bend right]    \\
                          & K \arrow[u, "\tau" description, bend right]
        \end{tikzcd}
\end{equation}

such that it is the right inverse of $\pi$
\begin{equation}
    \pi(\tau(k)) = k \text{ for all } k \in K 
\end{equation}

In a locally trivial fiber bundle, $E = K \times F$ \cite{rowlandFiberBundle,FiberBundle2020}:
\begin{equation}
\tau(k) = (k, g(k))
\end{equation}

where the domain of $g(k)$ is $F_k$ and returns a data point $r$. The space of all possible sections $\tau$ of $E$ is $\Gamma(E)$. All datasets $\tau \in \Gamma(E)$ have the same variables $F$ and connectivity $K$ but can have different values such that $\tau_{i}\neq\tau_{j}$.

\begin{figure}[ht]
    \includegraphics[width=.2\linewidth]{figures/sections/math/fiberbundle.png}
    \caption{write up some words here}
    \label{fig:fiberbundle}
\end{figure}

As illustrated by figure~\ref{fig:fiberbundle}, the vertical lines $F$ are the range of possible temperature values embedded in the total space $E$. The base space $K$ of the fiber bundle is a line because the data points $r$ in $E$ are on a space that is  continous in one dimension. 

\subsubsection{Base Space $K$}

\begin{figure}[ht]
    \includegraphics[width=0.4\linewidth]{figures/sections/math/k_qspace.png}
    \label{fig:kquote}
\end{figure}

$K$ is the quotient space of $E$, meaning it is the set of equivalence classes of elements $r$ in $E$ defined via the map $\pi: E \rightarrow K$ that sends each $r \in E$ to its equivalence class in $[r] \in K$ \cite{QuotientSpaceTopology2020,QuotientSpaceTopology2020}.

As shown in figure~\ref{fig:kquote}, the fibers $F$ divide $E$ into smaller spaces consisisting of $F$ and an open set neighborhood around $F$. This subdivision is projected down to the toplology $\mathcal{T}$

\begin{equation}
\mathcal{T}_k = \{U\subseteqq K: \{r \in E: [r] \in U\}\in \mathcal{T}_E\}
\end{equation}

where $[r] \in U$ is the point $k \in K$ with an open set surrounding it that has an open preimage in $E$ under the surjective map $\pi: r \rightarrow [r]$. 



\begin{figure}[ht]
    \includegraphics[width=0.2\linewidth]{figures/sections/math/temp_1k.png}
    %% add box around neighboring P and Map
    \label{fig:k_data}
\end{figure}

\begin{figure}[ht]
    \includegraphics[width=0.2\linewidth]{figures/sections/math/temp_2k.png}
\end{figure}

in figure~\ref{fig:k_data}, temperature is the only one data field in $r$ but the $K$ base spaces are different. subfig[1] is a timeseries, so the temperature in $r$ at time $t$ is dependent on the temperature in $r_{t-1}$ and the temperature in $r_{t+1}$ is dependent on  $r_t$; this connectivity is expressed as a one dimensional $K$ where $K$ is the number line. In the case of the map, every temperature in $r$ is dependent on its nearest neighbors on the plane, and one way to express this is by encoding $K$ as a plane. $K$ does not know the time or latitude or longitude of the point as those are metadata variables describing the $k$ rather than the value of $k$. The mapping $\tau: K \rightarrow E$ provides the binding between the key $k \in K$ and the value $r$ in $E$ \cite{munznerChDataAbstraction}.

\subsubsection{Fiber Space $F$}
We use Spivak's formalization of data base schemas as the basis of our fiber space $F$ \cite{spivakSIMPLICIALDATABASES}. He defines the type specification 
\begin{equation}
\pi: U \rightarrow DT
\end{equation}

where $DT$ is the set of data types (as identified by their names) and $U$ is the disjoint set of all possible objects $x$ of all types in $DT$. This means that for each type $T\in DT$, the preimage $\pi^{-1}(T)\subset U $ is the domain of $T$, and $x \in \pi^{1}(T)\subset U$ is an object of type $T$. Spivak then defines a schema $(C, \sigma)$ of type $\pi$, where $\pi$ is the universe of all types, such that 
\begin{equation}
\sigma: C \rightarrow DT
\end{equation}
where $C$ is the finite set of names of columuns, which we generalize to data fields in $E$. The set of all values restricted to the datatypes in $DT$ is $U_{\sigma}$

\begin{equation}
    \begin{tikzcd}
            U_{\sigma} \arrow[d, "\pi_{\sigma}" description] \arrow[r] & U \arrow[d, "\pi" description] \\
            C \arrow[r, "\sigma" description]                          & DT                            
    \end{tikzcd}
\end{equation}
The pullback $U_{\sigma} \coloneqq \sigma^{-1}(U)$ restricts $U$ to the datatypes of the fields in $C$ such that $U_{\sigma}$ is the fiber product $U \times_{DT} C$, and the pullback $\pi_{\sigma}:U_{\sigma} \rightarrow C$ specifies the domain bundle $U_{\sigma}$ over $C$ induced by $\sigma$. The fiber $F$ is the cartesian product of all sets in the disjoint union $U_{\sigma}$. 

For each field $c \in C$, the record function $r: C \rightarrow U_{\sigma}$ returns an object of type $\sigma(c) \in DT$. The set of all records $\Gamma(\sigma)$ is the set of all sections on $U_\sigma$. Spivak defines the $\tau$ mapping from an index of databases $K$ to records $\Gamma(\sigma)$ as $\tau: K \rightarrow \Gamma(\sigma)$. This is equivalent to $\tau: k \rightarrow E$ since $F = \Gamma(\sigma)$ and $F$ is the embedding in $E$ on which the records $r$ lie.
 

\begin{figure}[ht]
    \includegraphics[width=0.2\linewidth]{figures/sections/math/temp_2f.png}
    \label{fig:}
\end{figure}
\begin{figure}[ht]
    \includegraphics[width=0.2\linewidth]{figures/sections/math/temp_3f.png}
\end{figure}


%% pull out measurement
The fiber in figure~\ref{fig:temp} is the space of possible temperature values in degrees celsius, such that $F=[temp_{min}, temp_{max}]$ and is named \textrm{Temp}. In figure~\ref{fig:temp_time} \textrm{time} is encoded as a second dimension. This means that the set of possible values $F$ with $C=\{\textrm{Temp}, \textrm{Time}\}$:

\begin{equation}
F = [temp_{min}, temp_{max}] \times [time_{min}, time_{max}]
\end{equation}

and the function $\tau$ that retrieves records from $F$ is

\begin{equation}
\tau(k) =(k, (r: \textrm{Temp}\rightarrow temp, r: \textrm{Time}\rightarrow time))\\
temp \in [temp_{min}, temp_{max}], time \in [time_{min}, time_{max}])
\end{equation}

Since $\tau(k)=(k, r)$, $temp$ is bound to a named data field and $sigma$ binds $temp$ to a temperature data type. 

\subsubsection{Sheaf and Stalk}
As described in equation~\ref{eq:local_trivial}, there is a local space $U \subset K$ around every $k$. The inclusion map $\iota: U \rightarrow K$ can be pulled back such that $\iota^{*}E$ is the space of $E$ restricted over $U$. 
\begin{equation}
   
\end{equation}

The localized section of fibers $\iota^*\tau: U \rightarrow \iota^*E$ is the sheaf $O(E)$. The neighborhood of points the sheaf lies over is the stalk $\mathscr{F}_k$ \cite{StalkSheaf2019,spanier1989algebraic}

\begin{equation}
    \iota^{-1}\mathscr{F}(\{k\}) = \varinjlim_{k\subseteq U}\mathscr{F}(U) =  \varinjlim_{k \in U} = \mathscr{F}_k 
\end{equation}

which through $\iota$ gets the data in $E$ at and near to $k$. 


%not sure what's supposed to happen w/ sheafs here
% add U ^ V + intersection to get section on union which is condition of sheafs. 
\subsection{Prerender Space}
\label{sec:display}

Every point $k \in K$ maps to a space $S_{k} \in S$, which is the topology of the output of the artist $A$. The space $H$ is a total space representing the predisplay space, with a fiber dependent on the render space and a base space of $\S$:
\begin{equation}
    \begin{tikzcd}
        D \arrow[r, hook] & H \arrow[d, "\pi" description, bend right] \\
                                    & S \arrow[u, "\rho" description, bend right] 
    \end{tikzcd}
\end{equation}

where $\rho: H \rightarrow S$ is mapping from a mathematical encoding of the image to a region $xy$ on the screen that the renderer then maps to pixel space. 

For a physical screen display, the fiber $D$ encodes position and color of the visual elements as $\mathbb{R}^{7} = \{X, Y, Z, R, G, B, A\}$, where the patch $xy$ is a region in $H$ that we query into $S$ to get values that we integrate into an $\{r, g, b\}$ value. In the case of 2D screens, the predisplay space is a trivial fiber bundle $H=\mathbb{R}^{7}\times S$.

To draw an image, a region  $H$ is inverse mapped into a region $s \in S$ where $s$ is the inverse mapping from pixel to region on $s = \rho^{-1}_XY(xy)$ such that the rest of the fields in $\mathbb{R}^{7}$ are then integrated over $s$ to yield the remaining fields in $p$
\begin{align}
    R(p) &= \oint_s \rho_R(s)ds^{2}\\
    G(p) &= \oint_s \rho_G(s)ds^{2}\\
    B(p) &= \oint_s \rho_B(s)ds^{2}
\end{align}

Here we assume a single opaque 2D image such that the $z$ and $alpha$ fields can be omitted. 

\begin{figure}[h]
    \includegraphics[width=.4\linewidth]{figures/sections/math/render.png}
    \caption{}
    \label{fig:render}
\end{figure}

As illustrated in figure~\ref{fig:render}, words.

\subsection{Artist}

The artist is a mapping from the sheaf $O(E)$ which a dataset to a pre-render space $O(H)$ with display target properties $D$ and visualization structure $S$. 

\begin{equation}
    A: \Gamma(E) \rightarrow \Gamma(H)
\end{equation}

where $A$ is composed of an asthethic mapping $\Xi: \xi^*\tau \rightarrow \rho$ that builds the prerender render functions $\rho: S \rightarrow H$ described in section~\ref{sec:display}. Every region on $H$ maps back through $S$ to $K$ through $\xi: S_k \rightarrow K$ and $\xi^*k\tau$ pulls back to the data on $\xi^*E$ which is restricted over the visualization space $S$ . 

\begin{equation}
    \begin{tikzcd}
        F \arrow[r, hook] & E \arrow[d, "\pi" description]                           & \xi^*E \arrow[l] \arrow[d, "\pi" description]                                              &  & \xi^*E \arrow[rr, "\Xi" description] &                                                                                               & H \arrow[ld, "\rho^{-1}" description] \\
                          & K \arrow[u, "\tau" description, bend right, shift right] & S \arrow[l, "\xi" description] \arrow[u, "\xi^*\tau" description, bend right, shift right] &  &                                      & S \arrow[ru, "\rho" description, bend right, shift right] \arrow[lu, "\xi^*\tau" description] &                                      
        \end{tikzcd}
\end{equation}

The aesthetic property mapper $\Xi=Q\circ\nu$  builds $\rho$ functions via a function $Q$ that compose $\nu_{param}( \xi^*\tau)$ functions that map data in some field $c$ to a subset of $D$, such as x, or RGB. These $\rho$ functions are then mapped back into $S$ to provide the values for the region in $H$. $Q$ is analogous to a visualization type such as scatter or line plot, and $\nu$ to aesthetic parameters such as line color or marker shape \cite{bertinIIPropertiesGraphic2011, munznerMarksChannels}.

\begin{equation}
     \rho(u, v) = Q(\nu_(\tau_{c_0}), \ldots \nu_param_{n}(\tau_{c_n}))(u,v)
    \label{eq:Q}
\end{equation}

$Q$ composites $\nu$ functions because multiple parameters can target the same field in $D$, for example $\nu_{ypos}$ and $\nu_{linewidth}$ together determine the y position of a line drawn on screen such that $Q: \nu_{ypos}, \nu_{linewidth} \rightarrow \rho_{Y}$. 

\subsubsection{Marks}
Bertin describes a location on the plane as the signifying characteristic of a point, measurable length as the signifying characteristic of a line, and measurable size as the signifying characteristic of an area and that in display (pixel) space these are marks \cite{bertinIIPropertiesGraphic2011,carpendaleVisualRepresentationSemiology}. 
\begin{equation}
\begin{tikzcd}
    H \arrow[r, shift left] & S \arrow[l, "\rho(\xi^{-1}(J))", shift left] \arrow[rr, "\xi(s)", shift left] &  & J_{k} =  \{j \in K| \exists \Gamma \text{ s.t. } \Gamma(0)=k \text{ and }\Gamma(1)=j\} \arrow[ll, "\xi^{-1}(J)", shift left]
\end{tikzcd}
\label{eq:mark}
\end{equation}

Each point $s$ in the display space $H$, the mark it belongs to can be found by mapping $s$ back to $K$ via the lookup on S 

then taking $\xi(s)$ back to a point on $k \in K$ which lies on the connected component $J \subset K$. To got back to the display space $H$  from the simplacial complex $J$ of the signifier implanted in the mark, the inverse image of $J \in S, \xi^{-1}(J)$ is pushed back to $S$, and then  $\rho(\xi^{-1}(J))$ maps it into $R^{7}$. 


\subsubsection{Channels}
Visual channels are mappings from data into visual space, for example distinguishing categories by coloring them differently  \cite{bertinIIPropertiesGraphic2011,munznerMarksChannels}. Channels are an example of $\nu: c \mapsto d \in D$, which are the mappings from data $r \in \Gamma(\tau)$ to some part of the prerender fiber $D$. the $\nu$ functions can target only one field in $D$, such as $X$, or multiple such as $R,G,B$. 


% complex columns are combined to \nu 


%% \nu is structure preserving if  E->E  & H->H under nu. \phi can be one of the scales in Stevens

% concrete example of \nu as one of our encodings, if artist is faithfully preserving 
% if was fiber, trans on fiber bundle. we require that nu goes to visual side that can support same translation  - work through some examples : if an artist wanted to preserve, this is what it would have to do - for example relations, interval





%%The measurement spaces $X$ are each variables and have the properties of measurement scales, such as Steven's nominal, ordinal, interval, and ratio \cite{stevensTheoryScalesMeasurement1946}. Stevens talks about measurements and this is how we define it here:.... can check symmatry under stevens via carry through on tau

 Can in theory approximate hatching/dashing/etc can be approximated w/ functions and neighborhood of k. 


\subsubsection{Visual Idioms: Equivalance class of artists}
Two artists are equivalent when given data containers $\Gamma(E)$ of the same type, they output the same type of prerender $\Gamma(S)$:
\begin{equation}
    \begin{tikzcd}
        A_{\tau_2}: \arrow[d, shift right=2] & \Gamma(E) \arrow[r] & \Gamma(H) & \\
        A_{\tau_1}: \arrow[u, shift right]   & \Gamma(E) \arrow[r] & \Gamma(H) 
    \end{tikzcd}
\end{equation}

\end{document}