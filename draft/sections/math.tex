\documentclass[../main.tex]{subfiles}
\begin{document}

\section{Notation \& Definitions}
In this section we introduce a mathematical description of the visualization pipeline where artist $A$ functions transform data of type $\Gamma(E)$ to an intermediate representation in prerendered display space of type $\Gamma(H)$:

\begin{equation}
    A: \Gamma(E) \rightarrow \Gamma(H)
    \label{eq:artist}
\end{equation}

\begin{equation}
    A: \sigma \rightarrow \rho
\end{equation}

\begin{itemize}
\item $A$ is the function that converts an instance of data $\Gamma(E)$ to an instance of a visual representation $\Gamma(H)$ 
\item $E$ is a locally trivial fiber bundle over $K$ representing data space.
\item $K$ is a triangulizable space encoding the connectivity of the observations in the data. 
\item $H$ is a fiber bundle over $S$ representing visual space
\item $S$ is a simplacial complex of triangles encoding the connectivity of the visualization of $\Gamma(E)$
\end{itemize}

When $E$ is a trivial fiber bundle $E = F \times K$, it can be assumed that all $F_{k}$ for $k \in K$ are equal. Fiber bundles are product spaces of toplological spaces, which are a set of points with a set of neighborhoods for each point\cite{some topology text book, really wikipedia}. We will be discussing total spaces $E$ that are locally trivial, meaning that for every point in K there exists an open set neighborhood $U$ such that if you restrict $E$ to that point, then  $\iota*E = F\times U$. [there's a missing sentance here...]


\subsection{Data Model}

We use a fiber bundle model to represent the data, as proposed by Butler 
\cite{butlerVectorBundleClassesForm1992,butlerVisualizationModelBased1989}. A fiber bundle is a structure $(E, K, \pi, F)$  consisting of topological spaces $E, K, F$ and the map from total space to base space $\pi: E \rightarrow K$. 

\begin{tikzcd}
    F \arrow[r, hook] & E \arrow[d, "\pi" description, bend right ] \\
                      & K \arrow[u, description, bend right]
\end{tikzcd}


By definition, every point in the base space $k \in K$ has a local neighborhood $U$ such that \pi()
\cite{rowlandFiberBundle,FiberBundle2020}

\begin{tikzcd}
    \pi^{-1}(U) \arrow[r, "\varphi" description] \arrow[d, "\pi" description] & U \times F \arrow[ld, "\mathrm{proj}_U"] \\
    U                                                                         &                                         
\end{tikzcd}


There is a bijection from the fiber $F$ to the fiber $F_k$ over the point $k \in K$ such that 
\begin{equation}
\end{equation}


The function $\pi$ is the mapping from a point on a specific fiber $F_{k}|k\in K$ in $E$ to a location $k \in K$. The section $\sigma$ is the mapping from locations $k$ on $K$ to points on $F_{k}$ in $E$

\begin{tikzcd}
    F \arrow[r, hook] & E \arrow[d, "\pi" description, bend right ] \\
                      & K \arrow[u, "\sigma", description, bend right]
\end{tikzcd}

such that it is the right inverse of $\pi$:
\begin{equation}
    \pi(\sigma(k)) = k \text{ for all } \k \in K 
\end{equation}

In a locally trivial fiber bundle, $\sigma = K \times E$   \cite{weissteinFiberBundle}
\begin{equation}
\sigma(k) = (k, g(k))
\end{equation}

where the fiber at $k$, $F_k$,  is the domain of $g(k)$.  The space of global sections $\sigma$ of $E$ is $\Gamma(E)$:

\begin{figure}[ht]
    \label{fig:fiberbundle}
    \includegraphics[width=.2\linewidth]{figures/sections/math/fiberbundle.png}
    \caption{write up some words here}
    %% simpler still replace w/ timeseries of temp, 1D fiber -> inspired by butler 1989 FIgG 1
\end{figure}

As illustrated by figure~\ref{fig:fiberbundle}, the vertical lines $F$ are the range of possible temperature values embedded in the total space $E$. The base space $K$ of the fiber bundle describes the connectivity of the points in $E$; in figure~\ref{fig:fiberbundle} the connectivty of the timeseries is encoded in the line representation of $K$. 

\subsubsection{Base Space $K$}
\begin{figure}[ht]
    \includegraphics[width=0.2\linewidth]{figures/sections/math/temp_1k.png}
\end{figure}
\begin{figure}[ht]
    \includegraphics[width=0.2\linewidth]{figures/sections/math/temp_2k.png}
\end{figure}
%% general statement about topology, examples of 0, 1, ....Ks [none triangular things, square/torus, ....], condition on K is that it can be represent as a triangulation of K (can be realized as triangle on K - extra structure that may be arbitrary
%% k is a topological space ....equation soup 

K is a triangulizable topological space. 


\subsubsection{Fiber Space $F$}
The fiver ver is the set of elements $e \in F$ that lies over 
$F_{k}$ is a fiber over $K$ that lo 

Spivak provides a formalization for the values $x \in X$ whereby they can have specialized types. He defines the space of datatypes $\mathrm{DT}$ 

DT = set of strings names of types
U = set of single elements of type in DT
pi (U-DT)
Also need to take column names (U cross C) 

# C is more about naming types to reuse primatives 


Given $F = X_{0} \times X_{1} \times \ldots \times X_{n}$, where $X_{0}, X_{1}, \ldots X_{n}$ are measurement spaces (e.g. ints, floats, colors): 
\begin{equation}
g(k) = (x_0, x_1, \ldots x_n) \text{ where } x_0 \in X_0, x_1 \in X_1, \ldots x_n \in X_n
\end{equation}



\begin{figure}[ht]
    \includegraphics[width=0.2\linewidth]{figures/sections/math/temp_2f.png}
    \label{fig:}
\end{figure}
\begin{figure}[ht]
    \includegraphics[width=0.2\linewidth]{figures/sections/math/temp_3f.png}
\end{figure}
%%% might 
The fibers $F$ are a topological space embedded in $E$ on which lie the set of all possible values. For example, if $F$ is the interval $[0, 1]$, then $g(k)$ from equation~\ref{eq:sigma} returns a single measurement $x$ in the interval $F$:
\begin{equation}
    \label{eg:goff}
    g(k) = x, \text{ where } 0\leq x \leq 1
\end{equation}

The fiber in figure~\ref{fig:temp} is the space of possible temperature values in $\circ$ celsius, ranging from [start, end], similar to the interval $F$ in equation~\ref{eg:goff}. F can be any number of dimensions, for example in figure~\ref{fig:temp_time} time is encoded as a second dimension. Given:
\begin{itemize}
\item interval of all possible temperature values $[T_{min}, T_{max}]$ 
\item interval of all possible time values $[t_{min}, t_{max}]$
\end{itemize}

then $F$ is the cross product $F= [T_{min}, T_{max}] \times [t_{min}, t_{max}]$, and $g(k)$ listed in equation~\ref{eq:sigma} is:

\begin{equation}
g(k) = (x_0, x_1) \text{ where } x_0 \in [T_{min}, T_{max}], x_1 \in [t_{min}, t_{max}]
\end{equation}


% datetime type, columns = start, end 

% schema is ordered list of Column, type is approx F, U cross C over sigma, 
% space of rows/records f:k -> r, sigma:k -> r, is the table function (dataframe as instance)



\subsubsection{Subset}
$\Gamma(E)$ is the space of all points in $F$ returned by $\sigma$; therefore the points being visualized in a streaming or animation example can be considered a subset that lives on base space $U$ embedded in $K$ with the same fiber $\iota^*E$ and $\iota^*\sigma$.   

\begin{tikzcd}
    \iota^\ast E \arrow[r, hook] \arrow[d]                                                                       & E \arrow[d, bend right]                       \\
    U \arrow[r, "\iota" description, hook] \arrow[u, "\iota^\ast \sigma" description, bend right, shift right=2] & K \arrow[u, "\sigma" description, bend right]
\end{tikzcd}

\subsection{Prerender Space}
\label{sec:display}

% preamble type intro to displays, examples - displays can include screen, 3D printer, sphereical display, H is the total space of the screen. 11
\begin{figure}[h]
    \includegraphics[width=.4\linewidth]{figures/sections/math/render.png}
    \caption{}
    \label{fig:render}
\end{figure}

A physical display space can be thought of sets of $\mathbb{R}^{7}$ tuples, where 
\begin{equation}
    \mathbb{R}^{7} = \{X, Y, Z, R, G, B, A\}
\end{equation}

and the sets correspond to the sections on $\S$, which is the topology of the output of the artist $A$. The space $H$ is a total space representing the predisplay space, with a fiber of $\mathbb{R}^7$ and a base space of $\S$:

\begin{tikzcd}
    \mathbb{R}^{7} \arrow[r, hook] & H \arrow[d, "\pi" description, bend right] \\
                                   & S \arrow[u, "\rho" description, bend right] 
\end{tikzcd}


In the case of 2D screens, the predisplay space is a trivial fiber bundle $H=\mathbb{R}^{7}\times S$. As illustrated in figure~\ref{fig:render}, a region on the screen defined by the corners $(x_1, y_1)$ and $(x_2, y_2)$ maps into a region on a 2-simplex in $S$ defined by $(\alpha_1, \beta_1)$ and $(\alpha_2, \beta_2)$. The function on the simplex $f$ returns the (R, G, B, A) value for that $(\alpha, \beta)$ pair. For a region, 

\begin{equation*}
\rho(S) = \int_{\alpha_1}^{\alpha_2}\int_{\beta_1}^{\beta_2}\int_{z_1}^{z_2}{R, G, B, A}  
\end{equation*}

where the R,G,B,A values are derived from the how the data values are mapped to visual characteristics. The z component of the mapping to $\mathbb{R}^7$ is moved to the integration because this is a trivial space representing a 2D screen; $\rho$ varies depending on $H$. 

%%%pho is paramterized over alpha, beta, which is obtained via pullback lookup from (Hx,Hy)-?(alpha, beta)

\subsection{Artist}
%% include some words about motivation 
\begin{equation}
    A: \Gamma(E) \rightarrow \Gamma(H)
\end{equation}

\subsubsection{Screen to Data}
%%diagram: [data] -Tau->[RGVXYZ]
%%%            \ /\ / 
\begin{tikzcd}
    E \arrow[d, "\pi" description] & H \arrow[d, "\pi" description]                                                 \\
    K \arrow[u, "\sigma" description, bend right] & S \arrow[l, "\xi" description] \arrow[u, "\rho" description, bend right]
\end{tikzcd}

The pullback $\xi$ on $S \rightarrow K$ means that the values in $E$ can be directly mapped to a simplex in $S$, which means there's a mapping from screen space back to the values. 

\begin{tikzcd}
    \xi E \arrow[rr, "\tau" description] \arrow[rd, "\xi \sigma" description, bend right] &     & H \arrow[ld] \\
    & S \arrow[lu, bend right] &             
\end{tikzcd}


\subsubsection{Marks}
%% diagram of connected components/line thing 11-19-20 notes
Bertin describes a location on the plane as the signifying characteristic of a point, measurable length as the signifying characteristic of a line, and measurable size as the signifying characteristic of an area and that in display (pixel) space these are marks \cite{bertinIIPropertiesGraphic2011,carpendaleVisualRepresentationSemiology}. 
\begin{equation}
\begin{tikzcd}
    H \arrow[r, shift left] & S \arrow[l, "\rho(\xi^{-1}(J))", shift left] \arrow[rr, "\xi(s)", shift left] &  & J_{k} =  \{j \in K| \exists \Gamma \text{ s.t. } \Gamma(0)=k \text{ and }\Gamma(1)=j\} \arrow[ll, "\xi^{-1}(J)", shift left]
\end{tikzcd}
\label{eq:mark}
\end{equation}

Each point $s$ in the display space $H$, the mark it belongs to can be found by mapping $s$ back to $K$ via the lookup on S described in section~\ref{sec:display} then taking $\xi(s)$ back to a point on $k \in K$ which lies on the connected component $J \subset K$. To got back to the display space $H$  from the simplacial complex $J$ of the signifier implanted in the mark, the inverse image of $J \in S, \xi^{-1}(J)$ is pushed back to $S$, and then  $\rho(\xi^{-1}(J))$ maps it into $R^{7}$. 



\subsubsection{Visual Encodings}
Tau can preserves the measurement type properties (group scales)

%%The measurement spaces $X$ are each variables and have the properties of measurement scales, such as Steven's nominal, ordinal, interval, and ratio \cite{stevensTheoryScalesMeasurement1946}. Stevens talks about measurements and this is how we define it here:.... can check symmatry under stevens via carry through on tau

Tau is fully flexible and can do whatever; knows about fiber \& neighborhood of fiber. Can in theory approximate hatching/dashing/etc can be approximated w/ functions and neighborhood of k. 

%%hatches can either grow the R space or function 

\subsubsection{Visual Idioms: Equivalance class of artists}
Two artists are equivalent when given data containers $\Gamma(E)$ of the same type, they output the same type of prerender $\Gamma(S)$:
\begin{equation}
    \begin{tikzcd}
        A_{\tau_2}: \arrow[d, shift right=2] & \Gamma(E) \arrow[r] & \Gamma(H) &                                                \\
        A_{\tau_1}: \arrow[u, shift right]   & \Gamma(E) \arrow[r] & \Gamma(H) 
    \end{tikzcd}
\end{equation}

\end{document}