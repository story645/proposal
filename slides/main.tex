
\documentclass[xcolor={dvipsnames}, handout]{beamer}
\usepackage{amsmath,amsfonts,amssymb,pxfonts,eulervm,xspace}
\usepackage{mathrsfs} % math script fonts
\usepackage{tikz-cd} % commutative diagrams
\usepackage{subcaption} % subfigure float captioning

\usepackage{tabularx}
\usepackage{multirow}
\usepackage{graphicx}

\usepackage{notation} % move this later

\graphicspath{.figures/}

\usepackage[backend=bibtex, style=numeric-comp, doi=false,isbn=false,url=false, giveninits=true]{biblatex}
\bibliography{../draft/glasslab_viz.bib}

\usetheme{ccnycrest}

\newenvironment{changemargin}[2]{%
\begin{list}{}{%
\setlength{\topsep}{0pt}%
\setlength{\leftmargin}{#1}%
\setlength{\rightmargin}{#2}%
\setlength{\listparindent}{\parindent}%
\setlength{\itemindent}{\parindent}%
\setleng{}th{\parsep}{\parskip}%
}%
\item[]}{\end{list}}

\begin{document}

\title{Topological Artist Model}

\begin{frame}
	\titlepage
    Hannah Aizenman\\
    Advisor: Dr. Michael Grossberg \\
    Committee: Dr. Robert Haralick, Dr. Lev Manovich, Dr. Huy Vo\\
    External Member: Dr. Marcus Hanwell
\end{frame}

\section{Introduction}

\begin{frame}{What do visualization libraries do?}
    \begin{figure}
        \includegraphics[width=1\linewidth]{figures/intro/dar.png}
    \end{figure}
\end{frame}

\begin{frame}{Why are we doing this?}
    \begin{enumerate}
        \item visualization software design patterns are tuned to data structures\cite{HeerSoftware2006}
        \item visualization types tuned to data structures\cite{toryRethinkingVisualizationHighlevel2004}
        \item need to build a tool that supports domain specific structures and algorithms
        \item tool needs to be general purpose enough to support lots of domains
        \item tools needs to support 2D, 3D, static, dynamic, and interactive viz
    \end{enumerate}
\end{frame}

\subsection{Tools}
\begin{frame}{Everything is a table}
\begin{columns}
    \column{0.5\textwidth}
    \begin{figure}
        \includegraphics[width=1\textwidth]{figures/intro/grammar_example.png}
        \caption{Figure 1.5 in Wilkenson's Grammar of Graphics\cite{wilkinsonGrammarGraphics2005}}
    \end{figure}
    \column{0.5\textwidth}
        \begin{enumerate}
            \item  ggplot\cite{wickhamGgplot2ElegantGraphics2016a}, protovis\cite{bostockProtoviz2009}, D3 \cite{bostockDataDrivenDocuments2011}, vega\cite{satyanarayanDeclarativeInteractionDesign2014}, altair\cite{vanderplasAltairInteractiveStatistical2018}
            \item tends to be declarative\cite{heerDeclarative2010} 
            \item users describe how to compose visual elements \cite{wongsuphasawatNavigatingWideWorld2021}
        \end{enumerate}
    \end{columns}
    \pause
    \begin{center}
        \textbf{TAM: build the visual elements}
    \end{center}
\end{frame}

\begin{frame}{Everything is an image}
    \begin{columns}
        \column{0.5\textwidth}
        \begin{figure}
            \includegraphics[width=1\textwidth]{figures/intro/landsat.png}
        \end{figure}
        \column{0.5\textwidth}
        \begin{enumerate}
            \item ImageJ\cite{schneiderNIHImageImageJ2012}, ImagePlot\cite{studiesCulturevisImageplot2021}, Napari\cite{nicholas_sofroniew_2021_4533308}
            \item build plugins into existing system where image is primary input
        \end{enumerate}
    \end{columns}
    \pause
    \begin{center}
        \textbf{TAM: more general than tables and images}
    \end{center}
\end{frame}

\begin{frame}{Everything is ?}
    \begin{columns}
        \column{0.5\textwidth}
        \begin{figure}
            \includegraphics[width=1\textwidth]{figures/intro/dataset_diagram.png}
            \caption{Image is from the Data Representation chapter of the MayaVi 4.7.2 documentation.\cite{DataRepresentationMayavi}}
        \end{figure}
        \column{0.5\textwidth}
        \begin{enumerate}
            \item Matplotlib\cite{hunterMatplotlib2DGraphics2007},  VTK \cite{hanwellVisualizationToolkitVTK2015,geveci2012vtk}, MayaVi\cite{RamachandranMayaVI2011}, ParaView\cite{ahrens2005paraview}, Titan\cite{brianwylieUnifiedToolkitInformation2009}
            \item each visualization type API is tuned to the data structure
        \end{enumerate}
    \end{columns}
    \pause
    \begin{center}
        \textbf{TAM: common data abstraction interface between data and viz }
    \end{center}
\end{frame}

\subsection{Data}
\begin{frame}{Variables \& Topology}
    \begin{figure}
        \includegraphics[width=1\textwidth]{figures/intro/munzner_datatypes.png}
        \caption{Image is figure 2.8 in Munzner's Visualization Analysis and Design\cite{munznerVisualizationAnalysisDesign2014}}
    \end{figure}
    \begin{description}
        \item  metadata are \textit{keys} with associated \textit{values} (Munzner \cite{munznerVisualizationAnalysisDesign2014})
        \item[topology] Fiber bundles can be a common data abstraction (Butler \cite{butlerVectorBundleClassesForm1992,butlerVisualizationModelBased1989})
        \item[variables] Fiber as a database schema (Spivak \cite{spivakDatabasesAreCategories2010,spivakSIMPLICIALDATABASES})
    \end{description}
    \pause
    \begin{center}
        \textbf{Tam: variables are \textit{values} and \textit{keys} locate them in topology}
    \end{center}
\end{frame}

\subsection{Visualization}
\begin{frame}{Property Matching}
    \begin{figure}
        \includegraphics[width=.7\textwidth]{figures/intro/retinal_variables.png}
        \caption{This tabular form of Bertin's retinal variables is from Understanding Graphics \cite{malamedInformationDisplayTips2010} who reproduced it from Krygier and Wood's \textit{Making Maps: A Visual Guide to Map Design for GIS}\cite{krygierMakingMapsVisual2005}}
    \end{figure}
\end{frame}

\begin{frame}{Evaluating visualizations}
    \begin{description}
        \item[Expressiveness] structure preserving mappings from data to graphic (Mackinlay \cite{mackinlayAutomatingDesignGraphical1986})
        \item[Effectiveness] design choices made in deference to perceptual saliency \cite (Mackinlay \cite{clevelandResearchStatisticalGraphics1987,clevelandGraphicalPerceptionTheory1984,chambersGraphicalMethodsData1983a, munznerVisualizationAnalysisDesign2014})
        \item[Naturalness] easier to understand when properties match (Norman \cite{norman_things_smart})
        \item[Graphical Integrity] graphs show \textbf{only} the data (Tufte \cite{tufteVisualDisplayQuantitative2001})
    \end{description}
\end{frame}

\begin{frame}{Mathematical Frameworks for evaluating visualization}
    \begin{enumerate}
        \item APT: visualization has syntax and semantics like a language  (Mackinlay  \cite{mackinlayAutomatingDesignGraphical1986, mackinlayAUTOMATICDESIGNGRAPHICAL1987})
        \item Visualization has functional dependencies that can be represented as graphs (Sugibuchi \cite{sugibuchiFramwork2009}) 
        \item the semiotics of visualization are commutative in a category theory framework (Vickers \cite{vickersUnderstandingViz2013})
    \end{enumerate}
    \pause
    \begin{center}
        \textbf{Tam: Framework for building visualizations}
    \end{center}
\end{frame}

\begin{frame}{Visualization is commutative maps}
    data ($\alpha$) and viz ($\omega$) symmetries (Kindlmann and Scheidegger \cite{kindlmann2014algebraic})
        \begin{block}{$v\circ r_2 \circ \alpha =\omega\circ v\circ r_1$}
            \begin{equation*}
            \begin{tikzcd}[ampersand replacement=\&]
                D \arrow[d, "\alpha"'] \arrow[r, "r_1"] \& R \arrow[r, "\nu"]  \& V \arrow[d, "\omega"] \\
                D \arrow[r, "r_2"']                     \& R \arrow[r, "\nu"'] \& V                    
            \end{tikzcd}
        \end{equation*}
    \end{block}
    \pause
    \begin{center}
        \textbf{Tam: Add topology and make it functional}
    \end{center}
\end{frame}
\begin{frame}{So why?}
    \begin{figure}[H]
        \begin{subfigure}{.24\textwidth}
            \includegraphics[width=1\textwidth]{figures/intro/du_bois_spinny.png}
            \caption{}
            \label{fig:intro_dpa}
        \end{subfigure}
        \begin{subfigure}{.24\textwidth}
            \includegraphics[width=1\textwidth]{figures/intro/du_bois_line.png}
            \caption{}
            \label{fig:intro_dpb}
        \end{subfigure}
        \begin{subfigure}{.24\textwidth}
            \includegraphics[width=1\textwidth]{figures/intro/du_bois_country.png}
            \caption{}
            \label{fig:intro_dbc}
        \end{subfigure}
        \begin{subfigure}{.24\textwidth}
            \includegraphics[width=1\textwidth]{figures/intro/du_bois_heat.png}
            \caption{}
            \label{fig:intro_dbd}
        \end{subfigure}
        \caption{Du Bois' data portraits from the Prints and Photographs collection of the Library of Congress \cite{duboisGeorgiaNegroCity1900,duboisGeorgiaNegroValuation1900, duboisSeriesStatisticalCharts, duboisSeriesStatisticalChartsa}}
        \label{fig:intro_dubois}
    \end{figure}
    \begin{center}
        \textbf{Framework for building any valid visualization}
    \end{center}
\end{frame}

\subsection{Contributions}
\begin{frame}{Contributions}
    \begin{enumerate}
        \item a formal description of the topology preserving relationship between data and graphic via continuous maps
        \item a formal description of the property preservation from data component to visual representation as equivariant maps that carry a homomorphism of monoid actions
        \item abstraction of data structure using fiber bundles with schema like fibers to encode components and topology 
        \item algebraic sum operator such that more complex visualizations can be built from simple ones 
        \item a functional oriented visualization tool architecture built on the mathematical model to demonstrate the utility of the model
        \item a prototype of the architecture built on Matplotlib's infrastructure to demonstrate the feasibility of the model
    \end{enumerate}
\end{frame}

\section{Mathematical Framework}

\begin{frame}{Topological Artist Model}
    \begin{equation}
        \mathscr{\vartist}: \mathscr{\dtotal} \rightarrow \mathscr{\gtotal}
    \end{equation}
    \pause
    which is a map from data \dtotal\ to graphic \gtotal\ that carries a homomorphism of monoid actions 
    \pause 
    \begin{equation}
    \varphi: \monoid \rightarrow \monoid^{\prime}
    \end{equation}
    \pause
    such that artists are equivariant maps 
    \begin{equation}
    \mathscr{\vartist}(m\cdot \delement) = \varphi(m)\cdot\mathscr{\vartist}(\delement) 
    \end{equation}
\end{frame}

\subsection{Data Model}

\begin{frame}{Data Bundle}
    \begin{figure}
        \includegraphics[width=1\textwidth]{figures/math/fiberbundle.png}
    \end{figure}
    a fiber bundle is a tuple $(\dtotal,\,\dbase,\,\pi ,\,\dfiber)$ defined by the projection map $\pi$
    \begin{equation}
        \label{eq:fiber_bundle}
        \begin{tikzcd}[ampersand replacement=\&]
            \dfiber \arrow[r, hook] \& \dtotal \arrow[r, "\pi"] \& \dbase
        \end{tikzcd}
    \end{equation}
\end{frame}

\begin{frame}{Variables: Fiber}
    \begin{equation}
        \dfiber = \ftotal_{\fsection(\fname)} = \ftotal_{\ftype} 
    \end{equation}
    where $\ftotal_{\fsection(\fname)}$ is 
    \begin{equation}
        \label{eq:data_types}
        \begin{tikzcd}[ampersand replacement=\&]
            \fttype \arrow[r] \arrow[d, "\pi_{\fsection}"'] \& \ftotal \arrow[d, "\pi"] \\
            \fnames \arrow[r, "\fsection"']                  \& \ftypes       
        \end{tikzcd}
    \end{equation}
    \begin{description}
        \item[\ftypes] data types of the variables in the dataset 
        \item[\ftotal] disjoint union of all values of type $\ftype \in \ftypes$ 
        \item[\fnames] variable names
        \item[\fttype] \ftotal\ restricted to the data type of a named variable   
    \end{description}
\end{frame}

\begin{frame}{Sample Fiber}
    \begin{figure}[H]
        \begin{subfigure}{.4\textwidth}
            \includegraphics[width=\textwidth]{figures/math/temp_1k.png}
            \caption{}
            \label{fig:fiber_example_plane}
        \end{subfigure}
        \begin{subfigure}{.4\textwidth}
            \includegraphics[width=\textwidth]{figures/math/temp_3f.png}
            \caption{}
            \label{fig:fiber_example_cube}
        \end{subfigure}
        \caption{}
        \label{fig:data_fiber_example}
    \end{figure}
    \begin{description}
        \item[\ref{fig:fiber_example_plane}]  $\dfiber=\reals\times\reals$, for example (time, temperature)
        \item[\ref{fig:fiber_example_cube}]  $\reals\times\realsp\times\reals$, for example  (time, wind=(speed, direction))
    \end{description}
\end{frame}

\begin{frame}{Fiber Components}
    We can decouple decouple \dfiber\ into components $\dfiber_i$
    \begin{equation}
        \dfiber= \dfiber_{0} \times \ldots \times \dfiber_{i}\times\ldots\times \dfiber_{n}
    \end{equation}
\end{frame}

\begin{frame}{Structure of Components: Monoids}
A monoid \monoid\ is a set with
\begin{description}
    \item[associative binary operator] $\ast:\monoid \times \monoid\rightarrow \monoid$
    \item[identity element] $e\in \monoid$ such that $e\ast a= a \ast e = a$ for all $a \in \monoid$. 
\end{description}
\end{frame}
\begin{frame}{Monoid Actions}
A left monoid action of $\monoid_i$ is a set $\dfiber_i$ with an action $\bullet: \monoid\times \dfiber_i \rightarrow \dfiber_i$ with the properties:
    \begin{align*}
        \textbf{associativity}\;& \text{for all } f,g \in \monoid_i \text{ and } x\in \dfiber_i,\, f\bullet(g\bullet x) = (f\ast g) \bullet x\\
        \textbf{identity}\;& \text{for all } x\in \dfiber_i, e\in \monoid_i,\,  e\bullet x = x 
    \end{align*}
\end{frame}

\begin{frame}{Monoid Composition: Permutation}
    \begin{figure}
        \includegraphics[width=1\linewidth]{figures/math/monoid_emoji.png}
    \end{figure}
\end{frame}

\begin{frame}{Monoid Composition: Partial Order}
    MAKE THIS Hasse Diagram (borrow from theories of composablity)
\end{frame}

\begin{frame}{Continuity: base space}
    \begin{figure}[H]
    \includegraphics[width=\linewidth]{figures/math/k_qspace.png}
    \caption{}
    \label{fig:base_space_div}
\end{figure}
where the total space can be decomposed into components 
\begin{equation}
    \pi:\dtotal_1\oplus\ldots\oplus \dtotal_i \oplus\ldots \oplus \dtotal_n \rightarrow \dbase
\end{equation}
\end{frame}

\begin{frame}{Sample: Base Space}
    \begin{figure}[H]
        \begin{subfigure}{.4\textwidth}
            \includegraphics[width=\textwidth]{figures/math/temp_1k.png}
            \label{fig:base_example_discrete}
        \end{subfigure}
        \begin{subfigure}{.4\textwidth}
            \includegraphics[width=\textwidth]{figures/math/temp_2f.png}
            \label{fig:base_example_continuous}
        \end{subfigure}
        \label{fig:base_example}
    \end{figure}
\end{frame}

\begin{frame}{Values: Section}
    For any fiber bundle, there exists a map
    \begin{equation}
        \begin{tikzcd}[ampersand replacement=\&]
            \dfiber \arrow[r, hook] \& \dtotal \arrow[d, "\pi"'] \\
                              \& \dbase \arrow[u, "\dsection"', bend right]
        \end{tikzcd}
    \end{equation}
     such that $\pi(\dsection(\dbasepoint)) = \dbasepoint$. 
     
     The set of all global sections is denoted as $\Gamma(\dtotal)$.
\end{frame}

\begin{frame}{Record}
    Assuming a trivial fiber bundle $\dtotal = \dbase \times \dfiber$, the section is 
\begin{equation}
    \label{eq:section_return}
    \dsection(\dbasepoint) = (\dbasepoint, (g_{\dfiber_{0}}(\dbasepoint), \ldots, g_{\dfiber_{n}}(\dbasepoint)))
\end{equation}
where $g: \dbase \rightarrow \dfiber$ is the index function into the fiber.
\end{frame}

\begin{frame}{Sample dataset}
    \begin{figure}[H]
        \includegraphics[width=1\linewidth]{figures/math/fiberbundle.png}
        \label{fig:data_sections}
    \end{figure}
\end{frame}

\subsection{Graphics}
\begin{frame}{Fiber Bundle}
    The graphics bundle is a tuple $(\gtotal,\,\gbase,\,\pi ,\,\gfiber)$ defined by the projection map $\pi$
    \begin{equation}
        \label{eq:fiber_bundle}
        \begin{tikzcd}[ampersand replacement=\&]
            \gfiber \arrow[r, hook] \& \gtotal \arrow[r, "\pi"] \& \gbase
        \end{tikzcd}
    \end{equation}
\end{frame}


\begin{frame}{Continuity: Base space}
    The surjective map $\vindex: \gbase \rightarrow \dbase$ 
    \begin{equation}
        \begin{tikzcd}[ampersand replacement=\&]
            \dtotal \arrow[d, "\pi"'] \& \gtotal \arrow[d, "\pi"'] \\
            \dbase                   \& \gbase \arrow[l, "\vindex"']
        \end{tikzcd}
    \end{equation}
    goes from region $\gbasepoint \in \gbase_{\dbasepoint}$ to its associated point $\gbasepoint$. 
\end{frame}

\begin{frame}{Sample Display Target: Fiber}
    Assume a 2D opaque image $\gfiber=\reals^5$ with elements 
    \begin{equation}
    (x,\, y,\, r,\, g,\, b) \in \gfiber
    \end{equation}
    such that a rendered graphic only has 2D position and color.
\end{frame}

\begin{frame}{Continuity Retraction Maps}   
    \begin{figure}[H]
        \includegraphics[width=1\textwidth]{figures/math/retraction_maps.png}
        \label{fig:graphic_retraction_map}
    \end{figure}
\end{frame}

\begin{frame}{Rendering: Define a Pixel}
    \begin{columns}
        \column{0.5\textwidth}
        Given a pixel
        \begin{equation}
        p=\left[y_{top}, y_{bottom}, x_{right}, x_{left}\right]
        \end{equation}
        the inverse map of the bounding box 
        \begin{equation}
        \gbase_{p} ={\gsection_{xy}}^{-1}(p)
        \end{equation}
        is a region $\gbase_p \subset \gbase$ such that 
        \begin{align}
            \scriptstyle r_p &= \scriptstyle \iint\limits_{S_p} \rho_r(s)ds^{2}\\
            \scriptstyle g_p &= \scriptstyle \iint\limits_{S_p} \rho_g(s)ds^{2}\\
            \scriptstyle  b_p &= \scriptstyle \iint\limits_{S_p} \rho_b(s)ds^{2}
        \end{align}
        yields the color of the pixel. 
        \column{0.5\textwidth}
        \begin{figure}[H]
            \includegraphics[width=\textwidth]{figures/math/render.png}
        \end{figure}
    \end{columns}
\end{frame}

\begin{frame}[allowframebreaks]{References}
\printbibliography
\end{frame}
\end{document}

